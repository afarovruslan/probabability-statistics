\documentclass[a4paper,12pt]{article}

%%% Работа с русским языком
\usepackage{cmap}					% поиск в PDF
\usepackage{mathtext} 				% русские буквы в формулах
\usepackage[T2A]{fontenc}			% кодировка
\usepackage[utf8]{inputenc}			% кодировка исходного текста
\usepackage[english,russian]{babel}	% локализация и переносы
\usepackage{indentfirst}            % красная строка в первом абзаце
\usepackage{extarrows}              % длинное равно
\frenchspacing                      % равные пробелы между словами и предложениями

%%% Дополнительная работа с математикой
\usepackage{amsmath,amsfonts,amssymb,amsthm,mathtools} % пакеты AMS
\usepackage{icomma}                                    % "Умная" запятая
\usepackage{physics} % для символа нормы

%%% Свои символы и команды
\usepackage{centernot} % центрированное зачеркивание символа
\usepackage{stmaryrd}  % некоторые спецсимволы

\usepackage{pythonhighlight}

% \renewcommand{\epsilon}{\ensuremath{\varepsilon}}
% \renewcommand{\phi}{\ensuremath{\varphi}}
% \renewcommand{\kappa}{\ensuremath{\varkappa}}
% \renewcommand{\le}{\ensuremath{\leqslant}}
% \renewcommand{\leq}{\ensuremath{\leqslant}}
% \renewcommand{\ge}{\ensuremath{\geqslant}}
% \renewcommand{\geq}{\ensuremath{\geqslant}}
% \renewcommand{\emptyset}{\ensuremath{\varnothing}}

% \DeclareMathOperator*{\Mid}{\scalebox{1.1}{$\mid$}}
\DeclareMathOperator*{\argmax}{argmax}

% \DeclareMathOperator{\sgn}{sgn}
% \DeclareMathOperator{\gd}{\text{НОД}}
% \DeclareMathOperator{\lf}{\text{НОК}}
% \DeclareMathOperator{\rk}{rk}
% \DeclareMathOperator{\pr}{pr}
% \DeclareMathOperator{\im}{Im}
% \DeclareMathOperator{\ke}{Ker}
% \DeclareMathOperator{\re}{Re}
% \DeclareMathOperator{\cha}{char}
% \DeclareMathOperator{\ord}{ord}
% \DeclareMathOperator{\tr}{tr}
% \DeclareMathOperator{\md}{mod}
% \DeclareMathOperator{\Aut}{Aut}
% \DeclareMathOperator{\Inn}{Inn}
% \DeclareMathOperator{\End}{End}
% \DeclareMathOperator{\GL}{GL}
% \DeclareMathOperator{\SL}{SL}
% \DeclareMathOperator{\diag}{diag}

\newcommand{\divby}{
	\mathrel{\vbox{\baselineskip.65ex\lineskiplimit0pt\hbox{.}\hbox{.}\hbox{.}}}
}
\newcommand{\notdivby}{\centernot\divby}
% \newcommand\norm[1]{\left\lVert#1\right\rVert}
% \newcommand\normx[1]{\left\Vert#1\right\Vert}
\newcommand{\N}{\mathbb{N}}
\newcommand{\Z}{\mathbb{Z}}
\newcommand{\Q}{\mathbb{Q}}
\newcommand{\R}{\mathbb{R}}
\newcommand{\E}{\mathbb{E}}
\newcommand{\D}{\mathbb{D}}
\newcommand{\Cm}{\mathbb{C}}
\newcommand{\F}{\mathbb{F}}
\newcommand{\id}{\mathrm{id}}
\newcommand{\imp}[2]{(#1\,\,$\ra$\,\,#2)\,\,}
\newcommand{\nset}[1]{\{1, \dotsc, #1\}}
\newcommand{\Chi}{\scalebox{1.1}{\raisebox{\depth}{$\chi$}}}
\newcommand{\FF}{\scalebox{0.95}{$\mathcal F$}}
\newcommand{\FFF}{\scalebox{0.55}{$\mathcal F$}}
\newcommand{\GG}{\scalebox{0.95}{$\mathcal G$}}
\newcommand{\GGG}{\scalebox{0.55}{$\mathcal G$}}

\newcommand{\ND}[2]{\mathcal{N}\left({#1}, {#2}\right)} % normal distribution
\newcommand{\tod}{\xrightarrow[]{d}}
\newcommand{\toP}{\xrightarrow[]{P}}
\newcommand{\tolp}[1]{\xrightarrow[]{L_{#1}}}
\newcommand{\toac}{\xrightarrow[]{\text{п. н.}}} % almost certainly

\renewcommand\labelitemi{$\triangleright$}
\newcommand{\LL}{\mathcal{L}}
\newcommand{\todo}[1]{\textcolor{red}{TODO} #1}
\let\bs\backslash
\let\vect\overline
\let\normal\trianglelefteqslant
\let\lra\Leftrightarrow
\let\ra\Rightarrow
\let\la\Leftarrow
\let\gl\langle
\let\gr\rangle
\let\sd\leftthreetimes
\let\emb\hookrightarrow
\let\mc\mathcal
\let\mf\mathfrak

%%% Перенос знаков в формулах (по Львовскому)
\newcommand*{\hm}[1]{#1\nobreak\discretionary{}{\hbox{$\mathsurround=0pt #1$}}{}}
\renewcommand{\phi}{\varphi}
\newcommand{\eps}{\varepsilon}

%%% Работа с картинками
\usepackage{graphicx}    % Для вставки рисунков
\setlength\fboxsep{3pt}  % Отступ рамки \fbox{} от рисунка
\setlength\fboxrule{1pt} % Толщина линий рамки \fbox{}
\usepackage{wrapfig}     % Обтекание рисунков текстом

%%% Работа с таблицами
\usepackage{array,tabularx,tabulary,booktabs} % Дополнительная работа с таблицами
\usepackage{longtable}                        % Длинные таблицы
\usepackage{multirow}                         % Слияние строк в таблице

%%% Теоремы
\theoremstyle{definition}
\newtheorem{problem}{}
\newtheorem{theorem}{Теорема}
\newtheorem*{lemma}{Лемма}
% \newtheorem{proposition}{Утверждение}[section]
% \newtheorem*{exercise}{Упражнение}
% \newtheorem{problem}{}

% \theoremstyle{definition}
% \newtheorem{definition}{Определение}[section]
% \newtheorem*{corollary}{Следствие}
% \newtheorem*{note}{Замечание}
% \newtheorem*{reminder}{Напоминание}
% \newtheorem*{example}{Пример}

\theoremstyle{remark}
\newtheorem*{solution}{Решение}
\newtheorem*{guidance}{Указание}
\newtheorem*{prooff}{Д-во}

%%% Оформление страницы
\usepackage{extsizes}     % Возможность сделать 14-й шрифт
\usepackage{geometry}     % Простой способ задавать поля
\usepackage{setspace}     % Интерлиньяж
\usepackage{enumitem}     % Настройка окружений itemize и enumerate
\setlist{leftmargin=25pt} % Отступы в itemize и enumerate

\geometry{top=25mm}    % Поля сверху страницы
\geometry{bottom=30mm} % Поля снизу страницы
\geometry{left=20mm}   % Поля слева страницы
\geometry{right=20mm}  % Поля справа страницы

\setlength\parindent{15pt}                % Устанавливает длину красной строки 15pt
\linespread{1.3}                          % Коэффициент межстрочного интервала
%\setlength{\abovedisplayskip}{3pt}       % Отступы от выключных формул
%\setlength{\belowdisplayskip}{3pt}       % Отступы от выключных формул
%\setlength{\abovedisplayshortskip}{3pt}  % Отступы от выключных формул
%\setlength{\abovedisplayshortskip}{3pt}  % Отступы от выключных формул
%\setlength{\parskip}{0.5em}              % Вертикальный интервал между абзацами
%\setcounter{secnumdepth}{0}              % Отключение нумерации разделов
%\setcounter{section}{-1}                 % Нумерация секций с нуля
\usepackage{multicol}			          % Для текста в нескольких колонках
\usepackage{soulutf8}                     % Модификаторы начертания

%%% Содержаниие
\usepackage{tocloft}
\renewcommand{\thesection}{\arabic{section}.} 
\renewcommand{\thesubsection}{\thesection.\arabic{subsection}.}
\tocloftpagestyle{main}
%\setlength{\cftsecnumwidth}{2.3em}
%\renewcommand{\cftsecdotsep}{1}
%\renewcommand{\cftsecpresnum}{\hfill}
%\renewcommand{\cftsecaftersnum}{\quad}

%%% Шаблонная информация для титульного листа
\newcommand{\CourseName}{Математическая статистика}
\newcommand{\FullCourseName}{\so{МАТЕМАТИЧЕСКАЯ СТАТИСТИКА}}
\newcommand{\TaskNumber}{II}
\newcommand{\CourseDate}{весна 2024}
\newcommand{\AuthorInitials}{Яфаров Руслан}

%%% Колонтитулы
% \usepackage{titleps}
% \newpagestyle{main}{
% 	\setheadrule{0.4pt}
% 	\sethead{\CourseName}{}{\hyperlink{intro}{\;Назад к содержанию}}
% 	\setfootrule{0.4pt}                       
% 	\setfoot{ФПМИ МФТИ, \CourseDate}{}{\thepage} 
% }
% \pagestyle{main}  

%%% Нумерация уравнений
\makeatletter
\def\eqref{\@ifstar\@eqref\@@eqref}
\def\@eqref#1{\textup{\tagform@{\ref*{#1}}}}
\def\@@eqref#1{\textup{\tagform@{\ref{#1}}}}
\makeatother                      % \eqref* без гиперссылки
\numberwithin{equation}{section}  % Нумерация вида (номер_секции).(номер_уравнения)
\mathtoolsset{showonlyrefs=false} % Номера только у формул с \eqref{} в тексте.

%%% Гиперссылки
\usepackage{hyperref}
% \usepackage[usenames,dvipsnames,svgnames,table,rgb]{xcolor}
\hypersetup{
	unicode=true,            % русские буквы в раздела PDF
	colorlinks=true,       	 % Цветные ссылки вместо ссылок в рамках
	linkcolor=black!15!blue, % Внутренние ссылки
	citecolor=green,         % Ссылки на библиографию
	filecolor=magenta,       % Ссылки на файлы
	urlcolor=NavyBlue,       % Ссылки на URL
}

%%% Графика
\usepackage{tikz}        % Графический пакет tikz
\usepackage{tikz-cd}     % Коммутативные диаграммы
\usepackage{tkz-euclide} % Геометрия
\usepackage{stackengine} % Многострочные тексты в картинках

\begin{document}
	\begin{titlepage}
	\clearpage\thispagestyle{empty}
	\centering
	
	\textbf{Московский физико-технический институт}
	\vspace{33ex}
	
	{\textbf{\FullCourseName}}
	
	\TaskNumber\ ЗАДАНИЕ 
	\vspace{1ex}

	\begin{flushright}
		\noindent
		Автор: {\AuthorInitials},
		\\
		Б13-202 
	\end{flushright}
	
	\vfill
	\CourseDate
	\pagebreak
\end{titlepage}
	\section{Виды сходимости случайных векторов}
\problem{
Пусть $X_1,\dots, X_n$ — независимые одинаково распределенные случайные величины с распределением 
$Exp(\alpha),\ \alpha > 0$.\\
Рассмотрим статистику $Y=\frac{1}{n}\sum\limits_{i=1}^n X_i$.
Найдите такие константы $a(\alpha)$ и $\sigma^2(\alpha)>0$, что выполнено
$$
\sqrt{n}(Y\sin Y - a(\alpha))\xrightarrow[]{\text{d}}\mathcal{N}(0,\sigma^2(\alpha)),\text{ при }n\rightarrow \infty
$$
}
\solution{~
 По закону больших чисел $Y\tod\frac{1}{\alpha}$. По ЦПТ $\sqrt{n}\left(Y - \frac{1}{\alpha}\right) \tod \ND{0}{\frac{1}{\alpha^2}}$.
 Воспользуемся теоремой 1.4 из C2. Положим $h(x) = x\sin x, b_n = \frac{1}{\sqrt{n}}, \xi_n = \sqrt{n}\left(Y - \frac{1}{\alpha}\right), a = \frac{1}{\alpha}$. Тогда получим, что $\frac{h(a + \xi_nb_n) - h(a)}{b_n} = \sqrt{n}\left(Y\sin Y - \frac{1}{\alpha}\sin{\frac{1}{\alpha}}\right) \tod h'(a)\ND{0}{\frac{1}{\alpha^2}} \sim \left(\sin{\frac{1}{\alpha}} + \frac{1}{\alpha}\cos{\frac{1}{\alpha}}\right)\ND{0}{\frac{1}{\alpha^2}} \sim \ND{0}{\left(\frac{1}{\alpha}\sin{\frac{1}{\alpha}} + \frac{1}{\alpha^2}\cos{\frac{1}{\alpha}}\right)^2} \ra a(\alpha) = \frac{1}{\alpha}\sin{\frac{1}{\alpha}}, \sigma(\alpha) = \frac{1}{\alpha}\sin{\frac{1}{\alpha}} + \frac{1}{\alpha^2}\cos{\frac{1}{\alpha}}$
}
\problem{
Пусть $\{\xi_n\}_{n=1}^\infty,\{\eta_n\}_{n=1}^\infty,\{\zeta\}_{n=1}^\infty $- последовательности случайных величин.
 Докажите, что если $\xi_n\xrightarrow[]{d}\xi,|\xi_n-\eta_n|\leq\zeta_n|\xi_n|,\ \zeta_n\xrightarrow[]{P}0,$ то $\eta_n\xrightarrow[]{d}\xi$
 }
\solution{
	\lemma{
	Пусть $\xi_n$, посл-ть случайных величин, $\xi_n \ge 0, \xi_n \tod \xi$. Тогда 
	}
}
\problem{
Задан набор независимых одинаково распределённых случайных величин $X_1,\dots,X_n$ с распределением $\mathcal{N}(0,\sigma^2)$. 
Рассмотрим статистики $Y=\frac{1}{n}\sum\limits_{i=1}^n |X_i|,Z=\frac{1}{n}\sum\limits_{i=1}^n X_i^2$, $T=\sqrt{\frac{2}{\pi}}Z/Y$
Найдите предел сходимости по распределению выражения $\sqrt{n}(T-\sigma)$
}
\solution{~
	Найдем необходимые моменты и ковариации:
	\[\E|X_i| = \int_{\R} |x|p_{X_i}(x)dx = 2\int_{0}^{+\infty}x\frac{1}{\sqrt{2\pi\sigma^2}}e^{-\frac{x^2}{2\sigma^2}} dx = \frac{2}{\sqrt{2\pi\sigma^2}}\int_{0}^{+\infty} -\frac{\sigma^2}{x}xd(e^{-\frac{x^2}{2\sigma^2}}) = \left.\sigma\sqrt{\frac{2}{\pi}}e^{-\frac{x^2}{2\sigma^2}}\right\vert_{+\infty}^{0} \] \[= \sigma\sqrt{\frac{2}{\pi}}\]
	\[\E X_i^2 = \D X_i + (\E X_i)^2 = \sigma^2\]
	\[\E|X_i|X_i^2 = 2\int_{0}^{+\infty} x^3 \frac{1}{\sqrt{2\pi\sigma^2}}e^{-\frac{x^2}{2\sigma^2}}dx = -2\sigma^2\left(\left.x^2p_{X_i}(x)\right\vert_{0}^{+\infty} - 2\int_{0}^{+\infty}xp_{X_i}(x)dx\right) = 2\sigma^2 \E |X_i| = 2\sigma^3\sqrt{\frac{2}{\pi}}\]
	\[\E X_i^4 = -\sigma^2\left(0 - \int_{\R} p_{X_i}(x)3x^2dx \right) = 3\sigma^2\E X_i^2 = 3\sigma^4\]
	\[ \D |X_i| = \sigma^2 - \sigma^2 \frac{2}{\pi} = \sigma^2\frac{\pi - 2}{\pi}, \D X_i^2 = \E X_i^4 - (\E X_i^2)^2 = 3\sigma^4 - \sigma^4 = 2\sigma^4 \]
	\[ cov(|X_i|, X_i^2) = \E |X_i|X_i^2 - \E |X_i| \E X_i^2 = 2\sigma^3\sqrt{\frac{2}{\pi}} - \sigma\sqrt{\frac{2}{\pi}} \sigma^2 = \sigma^3\sqrt{\frac{2}{\pi}}\]
	Положим $\eta_i = (|X_i|, X_i^2)^T$, тогда $\D\eta_i = 
	\begin{pmatrix}
  \sigma^2\frac{\pi - 2}{\pi} & \sigma^3\sqrt{\frac{2}{\pi}} \\
	\sigma^3\sqrt{\frac{2}{\pi}} & 2\sigma^4
	\end{pmatrix}
	$
	Далее воспользуемся теоремой 1.4 для $\xi_n = \sum_{i = 1}^n \eta_i, h(x, y) = \sqrt{\frac{2}{\pi}}\frac{y}{x}, b_n = 1/\sqrt{n}, a = \E \eta_1 = (\sigma\sqrt{\frac{2}{\pi}}, \sigma^2)^T$.
	Получим \[\sqrt{n}(T - \sigma) \tod (\nabla h|_a, \xi) \sim \ND{0}{\nabla h|_a^T \D\eta \nabla h|_a}\]
	\[\nabla h = (-\sqrt{\frac{2}{\pi}}\frac{y}{x^2}, \sqrt{\frac{2}{\pi}}\frac{1}{x})^T \ra \nabla h|_a = (-\sqrt{\frac{\pi}{2}}, \frac{1}{\sigma})^T \ra \nabla h|_a^T \D\eta \nabla h|_a = \sigma^2 \left(\frac{\pi}{2} - 1\right)\]
}
\problem{
Пусть $\xi,\xi_1,\xi_2,\dots$ - такие случайные величины, что $(\xi_n-\xi)^2\xrightarrow[]{P}0\ $при $n\rightarrow\infty$
Показать, что $\xi_n^2\xrightarrow[]{P} \xi^2\ $при $n\rightarrow\infty$
}
\solution{
	$(\xi_n-\xi)^2\xrightarrow[]{P}0\ \lra P(|\xi_n - \xi|^2 > \eps^2) = P(|\xi_n - \xi| > \eps) \to 0 \ra \xi_n \toP \xi$. Далее пользуемся теоремой о наследовании сходимости для $h(x) = x^2$
}
\problem{
Пусть $\xi,\xi_1,\xi_2,\dots$ -  случайные величины. Привести пример, когда 
\begin{enumerate}
  \item $\xi_n\xrightarrow[]{L_2}\xi,\ \xi_n\centernot{\xrightarrow[]{\text{п.н.}}}\xi,\ n\rightarrow \infty$
  
  \item $\xi_n\xrightarrow[]{\text{п.н}}\xi,\ \xi_n\centernot{\xrightarrow[]{{L_2}}}\xi,\ n\rightarrow \infty$
  \item $\xi_n\xrightarrow[]{d}\xi,\ \xi_n\centernot{\xrightarrow[]{P}}\xi,\ n\rightarrow \infty$
\end{enumerate}
}
\solution{
	\begin{enumerate}
		\item 
		\item 
		\item Пусть $\xi \sim \ND{0}{1}, \xi_n = \xi$, тогда $\xi_n \tod -\xi$, но $P(|\xi_n + \xi| > \eps) = P(|\xi| > \eps / 2) \centernot{\xrightarrow[]{}} 0$
	\end{enumerate}
}
\problem{
Рассмотрим последовательность d-мерных случайных векторов $\overline{\xi}_n$.
Доказать, что если при некотором $\overline{c}\in\mathbb{R}^d$ выполнено соотношение 
$\overline{\xi}_n\xrightarrow[]{d}\overline{c},$ то $\overline{\xi}_n\xrightarrow[]{P}\overline{c}$
}
\solution{
	$\overline{\xi}_n \tod \overline{c} \ra \xi^i_n \tod c^i \forall i$. Тогда, если докажем, что $\xi \in \R, \xi \tod c \in \R \ra \xi \toP c$, то утверждение будет доказано, т. к. покомпонентная сх-ть по в-ти влечет сх-ть вектора.
	Пусть $\xi_n \in \R, \xi_n \tod c \in \R $. Тогда $P(|\xi_n - c| < \eps) = P(c - \eps < \xi_n < c + \eps) = F_{\xi_n}(c + \eps) - F_{\xi_n}(c - \eps + 0) \ge F_{\xi_n}(c + \eps) - F(c - \eps/2) \to 1 \ra \xi_n \toP c$
}
\section{Статистики и оценки. Построение и сравнение оценок}
\problem{
Пусть $X_1, \dots, X_n$ — выборка из распределения $R(0, \theta)$(равномерного распределения на отрезке $[0, \theta]$). Проверьте на несмещенность, состоятельность и сильную состоятельность следующие оценки параметра $\theta: 2\overline{X}, \overline{X} + X_{(n)}/2,
(n + 1)X_{(1)} , X_{(1)} + X_{(n)} , \frac{n + 1}{n}X_{(n)} $.
}
\solution{~
\begin{enumerate}
	\item 
   $\E (2\overline{X}) = 2\E X_1 = 2 \frac{\theta - 0}{2} = \theta \ra $ оценка несмещенна. \\
   По УЗБЧ $2\overline{X} \toac \theta \ra$ оценка сильно состоятельна.
   \item
   Пусть $X_{(n)} / 2 = \xi_n$. Найдем $p_{\xi_n}: F_{\xi_n}(x) = F_{X_1}^n(2x)$. $p_{\xi_n}(x) = \frac{d}{dx}F_{\xi_n}(x) = nF_{X_1}^{n - 1}(2x)p_{X_1}(2x) * 2 = n2^n \frac{x^{n - 1}}{\theta^n}I_{[0, \frac{\theta}{2}]}(x)$. $\E \xi_n = \int_{0}^{\theta/2} xn2^n \frac{x^{n - 1}}{\theta^n} dx = \left.n2^n\frac{x^{n + 1}}{\theta^n(n + 1)}\right\vert_{0}^{\frac{\theta}{2}} = \frac{n}{2n+ 2}\theta \ra \E(\overline{X} + X_{(n)} / 2) = \frac{\theta}{2}\left(1 + \frac{n}{n + 1}\right) \ne \theta \ra $ оценка является смещенной \\
   $F_{\xi_n}(x) =
   \begin{cases}
   1, x \ge \theta / 2, \\
   (\frac{2x}{\theta})^n, 0 < x < \theta / 2 \\
   0, x \le 0
   \end{cases}
   $, тогда при $n \to \infty$ получаем $F(x) = 
   \begin{cases}
   1, x \ge \theta / 2, \\
   0, x < \theta / 2 \\
   \end{cases}
    \ra \xi_n \tod \frac{\theta}{2} \lra \xi_n \toP \frac{\theta}{2} \ra $ по теореме о наследовании сходимости получаем, что $\overline{X} + \xi_n \toP \theta \ra $ оценка состоятельная \\
    Докажем, что $X_{(n)} \toac \theta$. Заф. $\omega \in \Omega$, тогда последовательность $\{X_{(n)}(\omega)\}$ является неубывающей и ограниченной сверху $\ra \exists \xi(\omega) : \lim_{n \to \infty} X_{(n)}(\omega) = \xi(\omega) \ra$ оценка сильно состоятельная, но $X_{(n)} \toP \theta \ra \xi(\omega) = \theta$. Аналогично доказывается, что $X_{(1)} \toac 0$. \\
    Тогда по теореме о наследовании сходимости оценка является сильно состоятельной

    \item Найдем $p_{(n + 1)X_{(1)}}: F_{X_{(1)}}(x) = 1 - (1 - F_{X_1}(x))^n \ra F_{(n + 1)X_{(1)}}(x) =  1 - (1 - F_{X_1}(\frac{x}{n + 1}))^n \ra p_{(n + 1)X_{(1)}}(x) = n\left(1 - \frac{x}{(n + 1)\theta}\right)^{n - 1} \frac{1}{\theta}I_{[0, (n + 1)\theta]}(x)$ \\
    $\E X_{(1)} = \int_{0}^{\theta}xn\left(1 - \frac{x}{\theta}\right)^{n - 1}\frac{1}{\theta}dx = \theta n\int_{0}^{1}t(1 - t)^{n - 1}dt = \theta nB(2, n) = \theta n\frac{\Gamma(2)\Gamma(n)}{\Gamma(n + 2)} = \theta n\frac{1!(n - 1)!}{(n + 1)!} = \frac{\theta}{n + 1} \ra $ оценка $X_{(1)}(n + 1) $ несмещенная. \\
    $F_{(n + 1)X_{(1)}}(x) = 
    \begin{cases}
     1, x \ge (n + 1)\theta, \\
     1 - \left(1 - \frac{x}{(n + 1)\theta}\right)^n, 0 < x < (n + 1)\theta \\
     0, x \le 0
    \end{cases}
    $ при $n \to \infty$ получим $
    F(x) =
    \begin{cases}
    1 - e^{-\frac{x}{\theta}}, x \ge 0, \\
    0, x < 0
    \end{cases} \ra 
    $ оценка не состоятельна и сл-но не сильно состоятельна
    \item $\E (X_{(1)} + X_{(n)}) = \frac{\theta}{n + 1} + \frac{\theta n}{n + 1} = \theta \ra $ оц-ка не смещена.
    Т. к. $F_{X_{(1)}} \to F(x) = 
    \begin{cases}
    1, x \ge 0 \\
    0, x < 0
    \end{cases}
    $, то $X_{(1)} \toP 0 \ra$ по теореме о наследовании сходимости оценка является сильно состоятельной.
    \item Оценка явлется несмещенной и  сильно состоятельной (по модулю предыдущих выкладок это очев).
  \end{enumerate}
}
\problem{
Пусть $\hat{\theta}_n(X)$ — асимптотически нормальная оценка параметра $\theta$ с асимптотической дисперсией $\sigma^2(\theta)$. Докажите, что тогда $\hat{\theta}_n(X)$ является состоятельной оценкой $\theta$.
}
\solution{
	Пусть $\xi_n = \frac{1}{\sqrt{n}}$. $\xi_n \toP 0$ По теореме о наследовании сходимости $\xi_n\sqrt{n}(\hat{\theta}_n(X) - \theta) \toP 0 \ra \hat{\theta}_n(X) \toP \theta$
}
\problem{
Пусть $X_1, \dots, X_n$ - выборка из распределения с параметром $\sigma^2$. Пусть, кроме того $D_{\sigma^2}X_1 = \sigma^2$ Докажите, что статистика $s = 1/n\sum_{i=1}^n (X_i - \overline{X})^2$ равна $\overline{X^2} - \left(\overline{X}\right)^2$ и является состоятельной оценкой $\sigma^2$. Является ли она несмещенной оценкой того же параметра?
}
\solution{
	$	1/n\sum_{i=1}^n (X_i - \overline{X})^2 = 1 / n \sum_{i = 1}^n (X_i^2 - 2X_i\overline{X} + \left(\overline{X}\right)^2) = 1 / n \sum_{i = 1}^n X_i^2 - 2\overline{X}1 / n \sum_{i = 1}^n X_i - \left(\overline{X}\right)^2 = \overline{X^2} - 2\left(\overline{X}\right)^2 + \left(\overline{X}\right)^2 = \overline{X^2} - \left(\overline{X}\right)^2$. Подставляя в равентсво вместо $X_i \text{  }X_i - \E X_i$ мы получим, что данная оценка равна $\overline{(X - \E X_1)^2} - (\overline{X} - \E X_1)^2 \toac \D X_1$ по УЗБЧ. Но $\E s = \D X_1 - \D \overline{X} = \D X_1 - \frac{1}{n}\D X_1 \ne \D X_1 \ra $ оценка смещена 
}
\problem{
Пусть $X_1, \dots, X_n$ — выборка из экспоненциального распределения с параметром $\theta$. Покажите, что $\forall k \in \N$ статистика $\sqrt[k]{k!/\overline{X^k}}$ является асимптотически нормальной оценкой параметра $\theta$. Найдите ее асимптотическую дисперсию.
}
\solution{
	\[\psi_{X_1}(t) = \E e^{itX_1} = \int_{0}^{+\infty} \theta e^{-\theta x} e^{itx}dx =  \int_{0}^{+\infty} \theta e^{x(it - \theta)}dx = -\frac{\theta}{it - \theta} = \left(1 - \frac{it}{\theta}\right)^{-1}\] \[ \frac{d^k}{dt^k}\psi_{X_1}(t) = \frac{i^k}{\theta^k}k!\left(1 - \frac{it}{\theta}\right)^{-k - 1} \ra \E X_1^k = \frac{k!}{\theta^k}\]. Пользуясь теоремой 1.4 из C2 для $h(x) = \sqrt[k]{k!/x}, \xi_n = \sqrt{n}\left(\overline{X^k} - \frac{k!}{\theta^k}\right), b_n = 1 /\sqrt{n}$ и $a = \frac{k!}{\theta^k}. h'(x) = (\sqrt[k]{k!}x^{-\frac{1}{k}})' = -\frac{1}{k}\sqrt[k]{\frac{k!}{x^{k + 1}}} \ra h'(a) = -\frac{\theta^{k + 1}}{k!k} \ra \sqrt{n}\left(\sqrt[k]{k!/\overline{X^k}} - \theta\right) \tod \ND{0}{\sigma^2(\theta)}$, где $\sigma^2(\theta) = \left(\frac{\theta^k}{kk!}\right)^2$
}
\problem{
Пусть $X_1 , \dots , X_n$ - выборка из распределения с плотностью
\[p_{\alpha, \beta}(x) = \frac{1}{\alpha}e^{(\beta - x)/\alpha}I_{[\beta, +\infty)}(x) \]
где $\theta = (\alpha, \beta)$ — двумерный параметр. Найдите для $\theta$ оценку максимального правдоподобия. Докажите, что полученная
для $\alpha$ оценка $\hat{\alpha}_n$ является асимптотически нормальной, и найдите ее асимптотическую дисперсию.
}
\solution{Пусть $\theta = (\alpha, \beta)$
  \[ \LL(x, \theta) = \frac{1}{\alpha^n}e^{\sum_{i = 1}^n \frac{\beta - x_i}{\alpha}} I_{[\beta, +\infty]}(x_1, \dots, x_n) = \frac{1}{\alpha^n}e^{\sum_{i = 1}^n \frac{\beta - x_i}{\alpha}} I_{[\beta, +\infty]}(\min(x_1, \dots, x_n))\]
  Для того, чтобы произведение было не 0, должно выплняться $\beta \le \min(x_1, \dots, x_n)$, в то же время $\LL(x, \theta) \xrightarrow[\beta]{} \max_{\beta} \lra \sum_{i = 1}^n \beta - x_i \xrightarrow[\beta]{} \max \ra \hat{\beta} = X_{(1)}$. $l(x, \theta) = \sum_{i = 1}^n \frac{\beta - x_i}{\alpha} - n\ln \alpha \ra \hat{\alpha_n} = \sum_{i = 1}^n\frac{X_i - \beta}{n} = \overline{X} - X_{(1)}$ в силу того, что функция $f(x) = \frac{c}{x} - \ln x, c \le 0$ имеет глобальный максимум в т. $x = -c$ 
}
\problem{
Найдите оценку максимального правдоподобия для параметра сдвига в распределении Коши, т.е. плотность равна
\[p_{\theta}(x) = \frac{1}{\pi(1 + (x - \theta)^2)}\]
если выборка состоит из а) одного наблюдения, б) двух на-
блюдений (т.е. $n = 1, 2$).
}
\solution{~
  \begin{enumerate}[label=\alph*]
  \item $\LL(x, \theta) = \frac{1}{1 + (x - \theta)^2} \ra \hat{\theta} = X_1$
  \item $l(x, \theta) = -\ln(1 + (x_1 - \theta)^2) - \ln(1 + (x_2 - \theta)^2) \ra \frac{dl}{dx} = \frac{(x_1 + x_2 - 2\theta)(\theta^2 - \theta x_1 - \theta x_2 + x_1 x_2 + 1)}{(1 + (x_1 - \theta)^2)(1 + (x_2 - \theta)^2)}$. Получаем 2 случая:
  \begin{enumerate}
    \item $|x_1 - x_2| \le 2$ \\
    Тогда уравенение правдоподобия имеет единственный корень $\frac{x_1 + x_2}{2}$. Легко заметить, что в этой точке достигается максимум всей функции
    \item $|x_1 - x_2| > 2$. Тогда уравнение имеет 3 корня и максимум достигается в одной из точек $\frac{x_1 + x_2}{2} + \sqrt{\left(\frac{x_1 - x_2}{2}\right)^2 - 1},\frac{x_1 + x_2}{2} - \sqrt{\left(\frac{x_1 - x_2}{2}\right)^2 - 1}$. Легко убедиться, что значение функции правдободобия совпадают на них
  \end{enumerate}
    \[ \text{Тогда оценка максимального правдободобия } \hat{\theta} = \\
    \begin{cases}
    \frac{X_1 + X_2}{2}, |X_1 - X_2| \le 2 \\
    \frac{X_1 + X_2}{2} + \sqrt{\left(\frac{X_1 - X_2}{2}\right)^2 - 1}, |X_1 - X_2| > 2
    \end{cases} \]
  \end{enumerate}
}
\problem{
Пусть $X_1 \sim R(0, \theta)$. Найдите несмещённую оценку параметра $1/\theta$.
}
\solution{
  Для $n = 1 \text{ } \E g(X_1) = \int_{0}^{\theta} \frac{1}{\theta} g(x) dx = \frac{1}{\theta} \ra \forall \theta \int_{0}^{\theta} g(x)dx = 1$, что невозможно. \\
  Для $n \ge 2$ пусть $\hat{\theta} = \frac{1}{4\sqrt{X_1X_2}}$. Тогда \[\E \hat{\theta} = \int_{[0, \theta]^2}\frac{1}{\theta^2} \frac{1}{2\sqrt{x_1}} \frac{1}{2\sqrt{x_2}} dx_1 dx_2 = \frac{1}{\theta}\]
}
\problem{
Найдите несмещенную оценку $\lambda^3$ по выборке $X_1, \dots, X_n$ из распределения $Pois(\lambda)$.
}
\solution{
  $\phi_{X_1}(t) = \E e^{tX_1} = e^{\lambda (e^t - 1)} \ra \frac{d}{dt}\phi_{X_1}(t) = \lambda e^{t + \lambda(e^t - 1)}, \frac{d^2}{dt^2}\phi_{X_1}(t) = \lambda e^{t + \lambda(e^t - 1)} + \lambda^2 e^{2t + \lambda(e^t - 1)}, \frac{d^3}{dt^3}\phi_{X_1}(t) = \lambda e^{t + \lambda(e^t - 1)} + \lambda^2 e^{2t + \lambda(e^t - 1)} + \lambda^2(2 + \lambda e^t)e^{2t + \lambda(e^t - 1)} \ra \E X_1 = \lambda, \E X_1^2 = \lambda + \lambda^2, \E X_1^3 = \lambda^3 + 3\lambda^2 + \lambda \ra \hat{\lambda} = X_1^3 - 3X_1^2 + 2X_1$
}
\problem{
Пусть $X_1, \dots, X_n$ — выборка из равномерного распределения на отрезке $[0, \theta]$. Сравните следующие оценки параметра $ \theta $ в равномерном подходе с квадратичной функцией потерь: $\theta: 2\overline{X},
(n + 1)X_{(1)}, \frac{n + 1}{n}X_{(n)} $.
}
\solution{
  Поскольку все оценки несмещены, фактически нужно сравнить дисперсии этих случайных величин.
  \begin{enumerate} 

    \item $\D (2 \overline{X}) = \frac{4}{n} \D X_1 =  \frac{\theta^2}{3}$.
    \item $\D (n + 1)X_{(1)} = (n + 1)^2 \D X_{(1)}. \E X_{(1)}^2 = \int_{0}^{\theta}x^2n\left(1 - \frac{x}{\theta}\right)^{n - 1}\frac{1}{\theta}dx = n\theta^2\int_{0}^{1}t^2(1 - t)^{n - 1}dt = \theta^2 nB(3, n) = \theta^2 n\frac{2!(n - 1)!}{(n + 2)!} = \frac{2\theta^2}{(n + 1)(n + 2)} \ra \D X_1 = \frac{2\theta^2}{(n + 1)(n + 2)} - \frac{\theta^2}{(n + 1)^2} = \frac{n\theta^2}{(n + 1)^2(n + 2)} \ra \\ \D (n + 1)X_{(1)} = \theta^2 \left(1 - \frac{2}{n + 2}\right)$
    \item $\E X_{(n)}^2 = \int_{0}^{\theta} nx^2 \frac{x^{n - 1}}{\theta^n}dx = \theta^2 \frac{n}{n + 2} \ra \D X_{(n)} = \theta^2n \left( \frac{1}{(n + 2)} - \frac{n}{(n + 1)^2}\right) = \theta^2 \frac{n}{(n + 1)^2(n + 2)} \ra \D \frac{n + 1}{n}X_{(n)} = \frac{\theta^2}{(n + 1)(n + 2)}$
  \end{enumerate}
  Таким образом, $3_{\text{оц}} \text{ лучше } 2_{\text{оц}} \text{ лучше } 1_{\text{оц}}$
}
\problem{
Пусть $\theta^*_1(X)$ и $\theta^*_2(X)$ — две наилучшие в среднеквадратичном подходе оценки параметра $\theta$ в классе всех оценок с од-
ним и тем же математическим ожиданием $\tau(\theta)$. Докажите, что тогда для любого $\theta$ они совпадают почти наверное.
}
\solution{
$\E(\theta^*_1(X) - \theta)^2 = \E(\theta^*_2(X) - \theta)^2$, при этом $\E \theta^*_1(X) = \E \theta^*_2(X) \ra \E (\theta^*_1(X))^2 = \E (\theta^*_2(X))^2 $
  . Предположим, $\theta^*_1(X) \text{ и } \theta^*_2(X)$ не совпадают почти наверное. Тогда $\E(\theta^*_1(X) - \theta^*_2(X))^2 > 0 \ra \E \theta^*_1(X) \theta^*_2(X) < \E (\theta^*_1(X))^2$.  Тогда \[\E(\theta^*_1(X) - \theta)^2 - \E \left(\frac{\theta^*_1(X) + \theta^*_2(X)}{2} - \theta\right)^2 = \E (\theta^*_1(X))^2 - \frac{\E (\theta^*_1(X))^2}{2} + \frac{\E \theta^*_1(X) \theta^*_2(X)}{2} =\] \[ \frac{1}{2}(\E (\theta^*_1(X))^2 - \E \theta^*_1(X) \theta^*_2(X)) > 0 \ra \]
  оценка $\frac{\theta^*_1(X) + \theta^*_2(X)}{2}$ лучше в данном классе.
}
\problem{
Пусть $X_1 , \dots, X_n$ - выборка из нормального распределения с параметрами $(a, \sigma^2)$. Найдите эффективную оценку
а) параметра a, если $\sigma$ известно;
б) параметра $\sigma^2$ , если a известно.
Вычислите информацию Фишера одного наблюдения в обоих случаях
}
\solution{~}
\problem{
Пусть $X_1, \dots, X_n$ — выборка из логистического распределения со сдвигом $\theta$, имеющего плотность
\[p_{\theta}(x) = \frac{\exp\{\theta - x\}}{(1 + \exp\{\theta - x\})^2}\]
Найдите информацию Фишера $i(\theta)$ одного наблюдения в этой модели.
}
\solution{
  \[u_{\theta}(x) = \frac{1 + e^{2\theta - 2x}}{(1 + e^{\theta - x})^2} \ra i(\theta) = \int_{\R} \frac{(1 + e^{2\theta - 2x})^2}{(1 + e^{\theta - x})^4} \frac{e^{\theta - x}}{(1 + e^{\theta - x})^2} dx = \int_{0}^{+\infty} \frac{(1 + t^2)^2}{(1 + t)^6} dt =\]
  \[ \int_{1}^{+\infty} \frac{(1 + (u - 1)^2)^2}{u^6}du = \int_{1}^{+\infty} \frac{u^4 - 4u^3+ 8u^2 - 8u + 4}{u^6}du =\]
  \[\left.\left(-1u^{-1} - 4 \frac{u^{-2}}{-2} + 8\frac{u^{-3}}{-3} - 8 \frac{u^{-4}}{-4} + 4 \frac{u^{-5}}{-5}\right)\right|_{1}^{+\infty} = -\left(-1+ 2 - \frac{8}{3} + 2- \frac{4}{5}\right) = \frac{7}{15}\]
}
\problem{
С помощью метода моментов построить оценку параметра $\theta$ для следующих распределений:
а)$Bern(\theta)$, б)$Pois(\theta)$, в)$\ND{\theta}{1}$, г)$\exp(\theta)$.
Является ли данная оценка:
\begin{enumerate}
  \item несмещенной?
  \item состоятельной?
  \item сильно состоятельной?
  \item асимптотически нормальной?
\end{enumerate}
}
\solution{
  В первых 3 случаях оценка $\hat{\theta} = \overline{X} $ является несмещенной, состоятельной и сильно состоятельной по УЗБЧ. По ЦПТ также она явлется асимптотически нормальной с асимптотической дисперсией равной дисперсии этой случайной величины, то есть 
  \begin{enumerate}[label=\alph*]
  \item $\D X_1 = \theta(1 - \theta)$
  \item $\D X_1 = \theta$
  \item $\D X_1 = 1$
  \item $\E X_1 = 1 / \theta \ra \hat{\theta} = \frac{1}{\overline{X}}$. По задаче 10 известно, что оценка асимптотически нормальная с ас-й дисперсией $\theta^2 \psi_{\overline{X}} = \psi_{X_1}^n(t/n) = \left(1 - \frac{it}{n\theta}\right)^{-n} \ra $
\end{enumerate}
}
\problem{
Решить предыдущую задачу использую вместо метода моментов метод максимального правдоподобия.
}
\solution{~}
\problem{
Рассмотрим распределения Коши с плотностью $p_{\theta}(x) = \frac{1}{\pi(1 + (x - \theta)^2)}$. С помощью выборочной медианы построить асимптотически нормальную оценку для $\theta^2$ и найти ее асимптотическую дисперсию.
}
\solution{~}
\problem{
Пусть $X_1, \dots, X_n$ — выборка из распределения:
а)$Bern(\theta)$, б)$Pois(\theta)$, в)$\ND{\theta}{1}$, г)$\exp(\theta)$.
Для каких функций существует эффективная оценка? Найти соответствующую эффективную оценку и количество информации (фишеровской), содежащейся в одном сообщении.
}
\solution{~}
\section{Достаточные статистики. Полные статистики. Оптимальные оценки. }
\problem{
Приведите пример такого параметрического семейства распределений $P$ и нетривиальной неполной достаточной статистики $S(X_1 , \dots, X_n)$, где $X_1 , \dots, X_n$ — выборка из неизвестного распределения $P \in \mathcal{P}$, что размерность статистики $S$ равна 1.
}
\solution{~}
\problem{
Пусть  $X_1 , \dots, X_n$ - выборка из нормального распределения
с параметрами $(a, \sigma^2)$, $a \in \R$, $\sigma > 0$. Найдите оптимальную оценку параметра $\theta = (a, \sigma^2)$.
}
\solution{~}
\problem{
Пусть $X_1 , \dots, X_n$ — выборка из нормального распределения с параметрами $(0, \theta^2)$. Найдите оптимальную оценку для $\theta$.
}
\solution{~}
\problem{
Пусть $X_1 , \dots, X_n$ — выборка из пуассоновского
распределения с параметром $\theta > 0$. Найдите $\E\left(X_1^2 \left\vert \displaystyle \sum_{i = 1}^n X_i \right.\right)$.
}
\solution{~}
\problem{
С помощью критерия факторизации найти достаточную статистику для следующего семейства распределений:
а)$Bern(\theta)$, б)$Pois(\theta)$, в)$\ND{\theta}{1}$, г)$\exp(\theta)$.
Проверить, является ли полученная статистика полной.
}
\solution{~}
\problem{
Построить оптимальную оценку функции $\tau(\theta) = 5\theta^2 + 3\theta + 7$ для $Bern(\theta)$.
}
\solution{~}
\problem{
Построить оптимальную оценку функции $\tau(\theta) = \sqrt{\theta}$ для $\exp(\theta)$
\end{document}